\documentclass{beamer}

\usepackage[utf8]{inputenc}
\usepackage[T1]{fontenc}
\usepackage[ngerman]{babel}
\usepackage{amsmath, amssymb}
\usepackage[babel,german=quotes]{csquotes}
\usepackage{tikz}
\usepackage{xcolor}
\usepackage{graphicx}
\usepackage{subcaption} % Add this in your preamble
\usepackage{fontawesome5}
\usepackage{comment}
\usepackage{hyperref}
\usepackage{changepage}

% \let\olditem\item
% \renewcommand{\item}{\olditem\vspace{0.3cm}}

\usepackage[sfdefault]{roboto}
\usefonttheme{professionalfonts}

% \usepackage[sfdefault]{sourcesanspro}
% \usefonttheme{professionalfonts}

% \usepackage[sfdefault,light]{sourcesanspro}
% \usefonttheme{professionalfonts}

\usetheme{Copenhagen}
% \usetheme{Frankfurt}
\setbeamertemplate{headline}{}
\setbeamertemplate{navigation symbols}{}
\setbeamercovered{transparent}

\title{\textbf{Marching Cubes}}
\subtitle{\small Ein Verfahren zur hochauflösenden 3D-Oberflächenrekonstruktion}
\author{Johannes Elsing}
\institute{Universität Freiburg}
\date{\today}

\setbeamercolor{author in head/foot}{bg=lightgray, fg=black}

\setbeamertemplate{footline}{
  \leavevmode%
  \hbox{%
    \begin{beamercolorbox}[wd=\paperwidth,ht=2.5ex,dp=1ex,leftskip=1em,rightskip=1em]{author in head/foot}%
      % \tiny Universität Freiburg -- Computer Science Department \hfill \insertshorttitle{} \hfill \insertframenumber{}/\inserttotalframenumber
      \tiny Universität Freiburg -- Institut für Informatik -- Computergrafik \hfill \insertframenumber{}/\inserttotalframenumber
    \end{beamercolorbox}
  }%
}

\begin{document}
\AtBeginSection[]{
  \begin{frame}
    \frametitle{Inhalt}
    \tableofcontents[currentsection]
  \end{frame}
}

\maketitle

\begin{frame}
  \frametitle{Inhalt}
  \tableofcontents
\end{frame}

\section{Einführung}
\begin{frame}[t]
  \frametitle{Einführung}
\begin{minipage}{\textwidth}
\includegraphics[scale=1]{Images/diagram.pdf}
\end{minipage}
\begin{center}
\includegraphics[scale=0.3]{Images/before.png}
\hspace{2.8cm}
\includegraphics[scale=0.3]{Images/after.png}
\end{center}

\vfill
\begin{columns}[t]
  \column{0.5\textwidth}
  \textbf{Eingabe}
  \begin{itemize}
    \item Kubische Gitterdaten (Voxel)
    \item Isowert
  \end{itemize}

  \column{0.5\textwidth}
  \textbf{Ausgabe}
  \begin{itemize}
    \item Isofläche
    \item Normalen der Eckpunkte
  \end{itemize}
\end{columns}

\end{frame}

\begin{frame}[t]
  \frametitle{Was ist Marching Cubes?}
  \centering
  % TikZ timeline
  \vspace{0.5cm}
  \begin{tikzpicture}[scale=0.8, every node/.style={font=\scriptsize}]
    \fill[blue!10] (-1,-1) rectangle (11.5,2);
    \draw[->, thick] (0,0) -- (10.5,0) node[right] {Zeit};

    \foreach \x/\year in {0/1987, 2/1991, 4/1995, 7/2004, 10/2021} {
      \draw[thick] (\x,0.15) -- (\x,-0.15);
      \node[align=center] at (\x,-0.5) {\year};
    }

    \node[align=center] at (0,0.9) {Marching\\Cubes};
    \node[align=center] at (2,0.6) {Asymptotic\\Decider};
    \node[align=center] at (2,1.5) {Marching\\Tetrahedra/};
    \node[align=center] at (4,0.9) {Marching\\Cubes 33};
    \node[align=center] at (7,0.9) {Dual\\Marching Cubes};
    \node[align=center] at (10,0.9) {Neural\\Marching Cubes};
  \end{tikzpicture}

  \vspace{-0.5cm}
  \includegraphics[width=\textwidth, trim=0 30 0 0, clip]{Images/scalar_field.pdf}

  \vspace{-0.5em}

  \begin{minipage}[t]{0.48\textwidth}
    \hspace*{2em}
    \raggedright
    Input: Skalarfeld
  \end{minipage}
  \hfill
  \begin{minipage}[t]{0.48\textwidth}
    \raggedleft
    Output: Triangulierte Isofläche
  \end{minipage}
\end{frame}

\section{Anwendungsbereiche}
\begin{frame}[t]
  \frametitle{Anwendungsbereiche von Voxeldaten}
  \begin{itemize}
    \item \textbf{Medizin:} CT-, MRT-Volumendaten
    \item \textbf{Ingenieurwesen:} Temperatur-/Drucksimulation
    \item \textbf{Strömungssimulation:} Raumdiskretisierung
    \item \textbf{Computergrafik:} Terrain, Wolken, visuelle Effekte
  \end{itemize}
  \vspace{0.4cm}
  \centering
  \includegraphics[scale=0.17]{Images/intersectedfoot.jpg}~\scriptsize[1]
  \includegraphics[scale=0.21]{Images/skull.png}~\scriptsize[2]
  \includegraphics[scale=0.1]{Images/terrain.jpg}~\scriptsize[3]
  \includegraphics[scale=0.3]{Images/tori.png}~\scriptsize[4]
\end{frame}

\section{Funktionsweise}
\begin{frame}[t]
  \frametitle{Wähle Zelle aus}
  \vspace{0.8cm}
  \hspace*{-0.8cm}\includegraphics[scale=1.2]{Images/select_cell_1.pdf}
\end{frame}

\begin{frame}[t]
  \frametitle{Klassifizierung}
  \vspace{0.8cm}
  \hspace*{-0.8cm}\includegraphics[scale=1.2]{Images/select_cell_2.pdf}
\end{frame}

\begin{frame}[t]
  \frametitle{Indexberechnung}
  \begin{center}
  \vspace{-0.5cm}
  \includegraphics[scale=2.2]{Images/cube_index.pdf}
  \vspace{-3cm}
  \begin{itemize}
    \item 8-Bit-Binärzahl (0–255)
  \end{itemize}
  \end{center}
\end{frame}

% \begin{frame}[t]
%  \frametitle{Grundprinzip: Isofläche}
%  \begin{itemize}
%    \item Ziel: Approximation einer Isofläche durch eine Linie innerhalb eines Gitters.
%    \item Isofläche (2D): $\forall a = (x, y) \in \mathbb{R}^2: f(a) = \rho$
%    \item Vergleiche Werte an Gitterpunkten $f(x, y)$ mit Schwellenwert $\rho$
%    \item Linie trennt Bereiche mit $f(x, y) < \rho$ und $f(x, y) > \rho$
%  \end{itemize}
% \end{frame}

\begin{frame}[t]
  \frametitle{Lookup \& Dreiecke}
  \begin{center}
  \makebox[\linewidth]{%
    \includegraphics[scale=0.35]{Images/Type-of-surface-combinations-for-the-marching-cube-algorithm.png}~\scriptsize[5]
  }
  \end{center}
  \begin{itemize}
    \item Für jede Konfiguration existiert eine Dreieckstopologie
    \item Nutzung einer Lookup-Tabelle
  \end{itemize}
\end{frame}

\begin{frame}[t]
  \frametitle{Kanteninterpolation}
  \begin{itemize}
    \item Schnittpunkte auf Kanten werden linear interpoliert.
    \item Glattere und realistischere Darstellung.
  \end{itemize}
  \vspace{0.5em}
  Für Eckpunkte $A$ und $B$ mit Werten $f_A$, $f_B$ und Isowert $\rho_{iso}$:
  \[
    t = \frac{\rho_{iso} - f_A}{f_B - f_A}, \quad
    \vec{P} = \vec{A} + t \cdot (\vec{B} - \vec{A})
  \]
  \vspace{0.5em}
\noindent
\begin{minipage}{\linewidth}
  \hspace{1cm}\includegraphics[scale=0.6]{Images/interpolation.pdf}
\end{minipage}
\end{frame}

\section{Normalenberechnung}
\begin{frame}
  \frametitle{Normalen aus Gradienten}
  \begin{itemize}
    \item Für realistisches Shading notwendig (Phong-Shading, Gouraud-Shading...)
  \end{itemize}
  \begin{center}
    \[
      G_x(i,j,k) = \frac{D(i+1, j, k) - D(i-1, j, k)}{2 \Delta x}
    \]
    \[
      G_y(i,j,k) = \frac{D(i, j+1, k) - D(i, j-1, k)}{2 \Delta y}
    \]
    \[
      G_z(i,j,k) = \frac{D(i, j, k+1) - D(i, j, k-1)}{2 \Delta z}
    \]
    \[
      \mathbf{n} = -\nabla D = - \begin{pmatrix} G_x \\ G_y \\ G_z \end{pmatrix}, \quad
      \hat{\mathbf{n}} = \frac{\mathbf{n}}{\|\mathbf{n}\|}
    \]
  \end{center}
\end{frame}

% Optimierung
\section{Optimierung}
\begin{frame}[t]
  \frametitle{Grenzen und Verbesserungen}
  \begin{itemize}
    \item leere Zellen
    \item Ambiguitäten bei bestimmten Konfigurationen
  \end{itemize}

  \begin{center}
  \includegraphics[scale=2.5]{Images/ambiguous_case.pdf}
  \end{center}

  \begin{itemize}
    \item Lösungsansätze:
    \begin{itemize}
      \item Asymptotic Decider
      \item Dual Marching Cubes
    \end{itemize}
  \end{itemize}
\end{frame}

\begin{frame}[t]
  \frametitle{Asymptotic Decider}
    \centering
    \[
    B(s, t) =
    (1 - s, s)
    \begin{pmatrix}
    B_{00} & B_{01} \\
    B_{10} & B_{11}
    \end{pmatrix}
    \begin{pmatrix}
    1 - t \\
    t
    \end{pmatrix}
    \]

\begin{columns}[c]

  \column{0.5\textwidth}
  \vspace{-10cm}
  \[
    \{(s, t) \in [0,1]^2 \mid B(s, t) = \rho_{\text{iso}}\}
  \]
  Ambiguität: Beide Hyperbeln schneiden $[0, 1]^2$

  \column{0.5\textwidth}
  \centering
  \vspace{0.2cm}
  \includegraphics[scale=0.5]{Images/asymptotic_decider.pdf}

\end{columns}

\end{frame}

\begin{frame}[t]
  \frametitle{Marching Tetrahedra}

  \begin{itemize}
    \item Würfel $\rightarrow$ 6 Tetraeder (gemeinsame Raumdiagonale)
    \item Keine topologischen Mehrdeutigkeiten
    \item 4 Ecken $\Rightarrow$ 16 Konfigurationen
  \end{itemize}

  \vspace{1ex}

  \hfill % schiebt alles nach rechts
  \begin{minipage}{0.9\textwidth}
    \begin{columns}[onlytextwidth]
      \column{0.45\textwidth}
        \includegraphics[scale=0.4]{Images/TetrasCases.png}~\scriptsize[6]
      \column{0.45\textwidth}
      \raggedleft
        \includegraphics[scale=0.1]{Images/cubesubdivisions.jpg}~\scriptsize[7]
    \end{columns}
  \end{minipage}
\end{frame}

\begin{frame}[t]
  \frametitle{Marching Cubes 33}

  \makebox[\linewidth][l]{
    \hspace*{-1cm}
    \begin{minipage}{0.95\linewidth}
      \begin{columns}[c]
        \column{0.55\textwidth}
        \parbox[t]{\linewidth}{
          \begin{itemize}
            \item Erweiterung des klassischen MC
            \item Vermeidung topologischer Fehler
            \item zusätzliche Konfigurationen (33 statt 15)
            \item Verwendet u.\,a. \textbf{Asymptotic Decider} zur Auswahl korrekter Triangulierungen
          \end{itemize}
        }

        \column{0.40\textwidth}
        \centering
        \hspace*{-0.45cm}
        \includegraphics[scale=0.13]{Images/MC33.png}~\scriptsize[8]\\
      \end{columns}
    \end{minipage}
  }
\end{frame}

\begin{frame}[t]
  \frametitle{Dual Marching Cubes}
  \begin{itemize}
    \item Pro Zelle im Octree ein Vertex im Dualgitter
    \item Vertex durch Minimierung einer QEF (Quadratische Fehlerfunktion) bestimmt
    \item QEF basiert auf Tangentialebenen der impliziten Funktion im Zellinneren
\end{itemize}
\includegraphics[scale=0.3]{Images/horse.png}~\scriptsize[9]
\hspace{1cm}
\includegraphics[scale=0.2]{Images/DualMarchingCubes.png}~\scriptsize[10]
\end{frame}

\begin{frame}[t]
  \frametitle{Neural Marching Cubes}
  \noindent
  \makebox[\linewidth]{%
    \includegraphics[scale=0.1]{Images/teaser.png}~\scriptsize[11]
    \hfill%
  }
  \begin{itemize}
    \item neuronales Netz
    \item geometrische Merkmale (scharfe Kanten und Ecken)
  \end{itemize}
\end{frame}

% Zusammenfassung
\section{Zusammenfassung}
\begin{frame}[t]
  \frametitle{Zusammenfassung Marching Cubes}
  \begin{itemize}
    \item Effizienter Algorithmus zur Isoflächen-Erzeugung aus volumetrischen Daten
    \item Einfach, schnell, weit verbreitet % in Medizin, Simulation und Grafik
    \item Herausforderungen:
    \begin{itemize}
      \item Ambiguitäten (topologische Inkonsistenzen)
      \item Hoher Speicherbedarf
    \end{itemize}
    \item Normale: aus Gradienten der Skalarwerte berechnet (z.\,B. zentrale Differenzen)
    \item Erweiterungen:
    \begin{itemize}
      \item \textbf{Asymptotic Decider}: vermeidet Ambiguitäten
      \item \textbf{Marching Tetrahedra}: vermeidet Ambiguitäten
      \item \textbf{MC 33}: topologisch korrekt
      \item \textbf{Dual MC}: bessere Meshqualität (Quads)
      \item \textbf{Neural MC}: lernbasierte Oberflächenapproximation
    \end{itemize}
  \end{itemize}
\end{frame}

\begin{frame}[t]
\frametitle{Bildquellen}
\tiny
\raggedright
[1] \url{https://www.cs.carleton.edu/cs_comps/0405/shape/smoothing.html}\par % foot
[2] \url{https://www.researchgate.net/figure/fig1_246088590}\par % Skull
[3] \url{https://www.youtube.com/watch?v=cAN0ClpPR48}\par % Terrain
[4] \url{https://github.com/BelfrySCAD/BOSL2/wiki/isosurface.scad}\par % Tori
[5] \url{https://www.researchgate.net/figure/Type-of-surface-combinations-for-the-marching-cube-algorithm-The-black-circles-means_fig2_282209849}\par
[6] \url{https://commons.wikimedia.org/wiki/File:TetrasCases.png}\par
[7] \url{https://github.com/andresbejarano/MarchingTetrahedra}\par % tetrahedra
[8] \url{https://www.cs.jhu.edu/~misha/ReadingSeminar/Papers/Chernyaev96.pdf}\par
[9] \url{https://www.cs.rice.edu/~jwarren/papers/dmc.pdf}\par % horse
[10] \url{https://hub.jmonkeyengine.org/t/jmonkey-dual-marching-cubes/25783}\par
[11] \url{https://arxiv.org/pdf/2106.11272}\par
\end{frame}

\begin{frame}[t]
\frametitle{Quellenangaben}
Lorensen, W. E., \& Cline, H. E. (1987). \textit{Marching Cubes: A High Resolution 3D Surface Construction Algorithm}. ACM SIGGRAPH Computer Graphics, 21(4), 163–169.\par
% Chen, Z., & Zhang, H. (2021). \textit{Neural Marching Cubes}\url{https://arxiv.org/abs/2106.11272}\par
https://de.wikipedia.org/wiki/Marching_Cubes
https://dl.acm.org/doi/10.5555/1025128.1026029
\end{frame}
\end{document}
