\documentclass{beamer}
\usepackage[utf8]{inputenc}
\usepackage[T1]{fontenc}
\usepackage[ngerman]{babel}
\usepackage{amsmath, amssymb}
\usepackage[babel,german=quotes]{csquotes}
\usepackage{media9}
\usepackage{tikz}
\usepackage{xcolor}
\usepackage{multimedia}
\usepackage{graphicx}
\usepackage{subcaption} % Add this in your preamble
\usepackage{fontawesome5}
\usepackage{comment}
\usepackage{hyperref}
\usepackage{changepage}
\usepackage{tcolorbox}
\usepackage{animate}

\tcbset{
  mygraybox/.style={
    colback=gray!15,  % sehr helles Grau
    colframe=white,   % kein Rahmen oder weiß
    boxrule=0pt,
    arc=2pt,
    left=2pt, right=2pt, top=2pt, bottom=2pt,
  }
}

\tcbset{highlightbox/.style={
  colback=gray!15,
  colframe=black,
  arc=2pt,
  boxrule=0.4pt,
  left=6pt,
  right=6pt,
  top=4pt,
  bottom=4pt
}}

\sloppy

% \let\olditem\item
% \renewcommand{\item}{\olditem\vspace{0.3cm}}

\usepackage[sfdefault]{roboto}
\usefonttheme{professionalfonts}

% \usepackage[sfdefault]{sourcesanspro}
% \usefonttheme{professionalfonts}

% \usepackage[sfdefault,light]{sourcesanspro}
% \usefonttheme{professionalfonts}

\usetheme{metropolis}
% \usetheme{Frankfurt}
\setbeamertemplate{headline}{}
\setbeamertemplate{navigation symbols}{}
\setbeamercovered{transparent}

\title{\textbf{Marching Cubes}}
\subtitle{\small Ein Verfahren zur hochauflösenden 3D-Oberflächenrekonstruktion}
\author{Johannes Elsing}
\institute{Universität Freiburg}
\date{\today}

\setbeamercolor{author in head/foot}{bg=lightgray, fg=black}

\setbeamertemplate{footline}{
  \leavevmode%
  \hbox{%
    \begin{beamercolorbox}[wd=\paperwidth,ht=2.5ex,dp=1ex,leftskip=1em,rightskip=1em]{author in head/foot}%
      % \tiny Universität Freiburg -- Computer Science Department \hfill \insertshorttitle{} \hfill \insertframenumber{}/\inserttotalframenumber
      \tiny Universität Freiburg -- Institut für Informatik -- Computergrafik \hfill \insertframenumber{}/\inserttotalframenumber
    \end{beamercolorbox}
  }%
}

\begin{document}
\AtBeginSection[]{
  \begin{frame}
    \frametitle{Inhalt}
    \tableofcontents[currentsection]
  \end{frame}
}

\maketitle

\begin{frame}
  \frametitle{Inhalt}
  \tableofcontents
\end{frame}

\section{Einführung}
\begin{frame}[t]
  \frametitle{Einführung}
\begin{minipage}{\textwidth}
\includegraphics[scale=1]{Images/diagram.pdf}
\end{minipage}
\begin{center}
\includegraphics[scale=0.3]{Images/before.png}
\hspace{2.8cm}
\includegraphics[scale=0.3]{Images/after.png}
\end{center}

\vfill
\begin{columns}[t]
  \column{0.5\textwidth}
  \textbf{\hspace{0.5cm} Eingabe}\\
  \hspace{0.5cm} Kubische Gitterdaten\\
  \hspace{0.5cm} Isowert\\

  \column{0.5\textwidth}
  \textbf{\hspace{1.1cm} Ausgabe}\\
  \hspace{1.2cm}Isofläche\\
  \hspace{1.2cm}Normalen der Eckpunkte\\
\end{columns}

\end{frame}

\begin{frame}[t]
  \frametitle{Was ist Marching Cubes?}
  \centering
  % TikZ timeline
  \vspace{0.5cm}
  \begin{tikzpicture}[scale=0.8, every node/.style={font=\scriptsize}]
    \fill[blue!10] (-1,-1) rectangle (11.5,2);
    \draw[->, thick] (0,0) -- (10.5,0) node[right] {Zeit};

    \foreach \x/\year in {0/1987, 2/1991, 4/1995, 7/2004, 10/2021} {
      \draw[thick] (\x,0.15) -- (\x,-0.15);
      \node[align=center] at (\x,-0.5) {\year};
    }

    \node[align=center] at (0,0.9) {Marching\\Cubes};
    \node[align=center] at (2,0.6) {Asymptotic\\Decider};
    \node[align=center] at (2,1.5) {Marching\\Tetrahedra/};
    \node[align=center] at (4,0.9) {Marching\\Cubes 33};
    \node[align=center] at (7,0.9) {Dual\\Marching Cubes};
    \node[align=center] at (10,0.9) {Neural\\Marching Cubes};
  \end{tikzpicture}

  \vspace{-0.5cm}
  \includegraphics[width=\textwidth, trim=0 30 0 0, clip]{Images/scalar_field.pdf}

  \vspace{-0.5em}

  \begin{minipage}[t]{0.48\textwidth}
    \hspace*{2em}
    \raggedright
    Input: Skalarfeld
  \end{minipage}
  \hfill
  \begin{minipage}[t]{0.48\textwidth}
    \raggedleft
    Output: Triangulierte Isofläche
  \end{minipage}
\end{frame}

\section{Anwendungsbereiche}
\begin{frame}[t]
  \frametitle{Anwendungsbereiche von Voxeldaten}
  \begin{itemize}
    \item \textbf{Medizin:} Visualisierung von CT- und MRT-Volumendaten
    \item \textbf{Computergrafik:} Terrain, Wolken, volumetrische Effekte
    \item \textbf{Modellierung:} implizite Oberflächen
  \end{itemize}
  \centering
  \includegraphics[scale=0.17]{Images/intersectedfoot.jpg}~\scriptsize[1]
  \includegraphics[scale=0.21]{Images/skull.png}~\scriptsize[2]
  \includegraphics[scale=0.1]{Images/terrain.jpg}~\scriptsize[3]
  \includegraphics[scale=0.5]{Images/torus.png}~\scriptsize[4]
\end{frame}

\section{Funktionsweise}
\begin{frame}[t]
  \frametitle{Marschiere Zelle für Zelle}
  \vspace{0.8cm}
  \hspace*{-0.8cm}\includegraphics[scale=1.2]{Images/select_cell_1.pdf}
\end{frame}

\begin{frame}[t]
  \frametitle{Klassifizierung der Zelle}
  \begin{itemize}
  \item \( f(v) > \rho_{\text{iso}} \) (innerhalb der Fläche)
  \item \( f(v) \leq \rho_{\text{iso}} \) (außerhalb der Fläche)
  \end{itemize}
  \hspace*{-0.8cm}\includegraphics[scale=1.2]{Images/select_cell_2.pdf}
\end{frame}

\begin{frame}[t]
  \frametitle{Topologische Konfigurationen}
  \begin{center}
  \makebox[\linewidth]{%
    \includegraphics[scale=0.35]{Images/Type-of-surface-combinations-for-the-marching-cube-algorithm.png}~\scriptsize[5]
  }
  \end{center}
  \begin{itemize}
    \item Für jede Konfiguration existiert eine Dreieckstopologie
    \item Nutzung einer Lookup-Tabelle
  \end{itemize}
\end{frame}

\begin{frame}[t]
  \frametitle{Indexberechnung}
  \begin{center}
  \vspace{-0.5cm}
  \includegraphics[scale=2.2]{Images/cube_index.pdf}
  \end{center}
\end{frame}

\begin{frame}[t]
  \frametitle{Indexberechnung}
  \begin{center}
  \vspace{-0.5cm}
  \includegraphics[scale=2.2]{Images/cube_index_2.pdf}
  \end{center}
\end{frame}

\begin{frame}[t]
  \frametitle{Topologische Konfigurationen}
  \begin{center}
    \includegraphics[scale=0.31]{Images/modified_head.pdf}~\scriptsize[6]
  \end{center}
\end{frame}

\begin{frame}[t]
  \frametitle{Kanteninterpolation}
  \begin{itemize}
    \item Schnittpunkte auf Kanten werden linear interpoliert.
    \item Glattere und realistischere Darstellung.
  \end{itemize}
  \vspace{0.5em}

\begin{tcolorbox}[mygraybox]
  Für Eckpunkte $A$ und $B$ mit Werten $f_A$, $f_B$ und Isowert $\rho_{iso}$:
  \[
    t = \frac{\rho_{iso} - f_A}{f_B - f_A}, \quad
    \vec{P} = \vec{A} + t \cdot (\vec{B} - \vec{A})
  \]
\end{tcolorbox}
\noindent
\begin{minipage}{\linewidth}
  \hspace{1cm}\includegraphics[scale=0.6]{Images/interpolation.pdf}
\end{minipage}
\end{frame}

\begin{frame}[t]
  \frametitle{Kanteninterpolation}
  \begin{tcolorbox}[highlightbox]
    \centering\textbf{Beispiel: Vergleich mit und ohne Kanteninterpolation}
  \end{tcolorbox}

  \vspace{0.5em}

  \begin{minipage}{0.48\textwidth}
    \centering
    \includegraphics[width=\linewidth]{Images/with_interpolation.png}
    \captionof{figure}{Mit Interpolation}
  \end{minipage}
  \hfill
  \begin{minipage}{0.48\textwidth}
    \centering
    \includegraphics[width=\linewidth]{Images/without_interpolation.png}
    \captionof{figure}{Ohne Interpolation}
  \end{minipage}

\end{frame}

\section{Normalenberechnung}
\begin{frame}
  \frametitle{Normalen aus Gradienten}
  \begin{itemize}
    \item Für realistisches Shading notwendig (z.B. Phong)
    \item Tangentialebene für Dual Marching Cubes
  \end{itemize}

  \begin{center}
    \fbox{%
      \parbox{0.8\linewidth}{%
        \[
          \mathbf{n} = -\nabla D = - \begin{pmatrix} G_x \\ G_y \\ G_z \end{pmatrix}, \quad
          \hat{\mathbf{n}} = \frac{\mathbf{n}}{\|\mathbf{n}\|}
        \]
        \vspace*{0.5cm}
        \[
          G_x(i,j,k) \approx \frac{D(i+1, j, k) - D(i-1, j, k)}{2 \Delta x}
        \]
        \[
          G_y(i,j,k) \approx \frac{D(i, j+1, k) - D(i, j-1, k)}{2 \Delta y}
        \]
        \[
          G_z(i,j,k) \approx \frac{D(i, j, k+1) - D(i, j, k-1)}{2 \Delta z}
        \]
      }
    }
\end{center}
  
\end{frame}

% Optimierung
\section{Optimierung}
\begin{frame}[t]
  \frametitle{Grenzen und Verbesserungen}
  \begin{itemize}
    \item leere Zellen
    \item Ambiguitäten bei bestimmten Konfigurationen:
  \end{itemize}

  \begin{center}
  \includegraphics[scale=2.5]{Images/ambiguous_case.pdf}
  \end{center}

  \begin{itemize}
    \item Lösungsansätze:
    \begin{itemize}
      \item Asymptotic Decider
      \item Dual Marching Cubes
    \end{itemize}
  \end{itemize}
\end{frame}

\begin{frame}[t]
  \frametitle{Asymptotic Decider}
    \centering
    \[
    B(s, t) =
    (1 - s, s)
    \begin{pmatrix}
    B_{00} & B_{01} \\
    B_{10} & B_{11}
    \end{pmatrix}
    \begin{pmatrix}
    1 - t \\
    t
    \end{pmatrix}
    \]

\begin{columns}[c]

  \column{0.5\textwidth}
  \vspace{-10cm}
  \[
    \{(s, t) \in [0,1]^2 \mid B(s, t) = \rho_{\text{iso}}\}
  \]
  Ambiguität: Beide Hyperbeln schneiden $[0, 1]^2$

  \column{0.5\textwidth}
  \centering
  \vspace{0.2cm}
  \includegraphics[scale=0.5]{Images/asymptotic_decider.pdf}

\end{columns}
\end{frame}

\begin{frame}[t]
  \frametitle{Asymptotic Decider}
  \small
  \begin{tabbing}
if $\rho_{\text{iso}} > B(S_\rho, T_\rho)$ then \\
\hspace{4ex} connect $(S_1, 1)$ to $(1, T_1)$ \\
\hspace{4ex} connect $(S_0, 0)$ to $(0, T_0)$ \\
else \\
\hspace{4ex} connect $(S_1, 1)$ to $(0, T_0)$ \\
\hspace{4ex} connect $(S_0, 0)$ to $(1, T_1)$
  \end{tabbing}

  \centering
  \includegraphics[scale=0.8]{Images/ambiguities.pdf}
\end{frame}

\begin{frame}[t]
  \frametitle{Marching Tetrahedra}

  \begin{itemize}
    \item Würfel $\rightarrow$ 6 Tetraeder (gemeinsame Raumdiagonale)
    \item Keine topologischen Mehrdeutigkeiten
    \item 4 Ecken $\Rightarrow$ 16 Konfigurationen
  \end{itemize}

  \vspace{1ex}

  \hfill % schiebt alles nach rechts
  \begin{minipage}{0.9\textwidth}
    \begin{columns}[onlytextwidth]
      \column{0.45\textwidth}
        \includegraphics[scale=0.4]{Images/TetrasCases.png}~\scriptsize[7]
      \column{0.45\textwidth}
      \raggedleft
        \includegraphics[scale=0.1]{Images/cubesubdivisions.jpg}~\scriptsize[8]
    \end{columns}
  \end{minipage}
\end{frame}

\begin{comment}
\begin{frame}[t]
  \frametitle{Marching Cubes 33}

  \begin{columns}[t]
    \begin{column}{2.7cm}
      \raisebox{4cm}[0pt][0pt]{% shift content UP by 3cm
        \makebox[0pt][l]{%
          \hspace*{-1.8cm}
          \begin{minipage}{6cm}
            \begin{itemize}
            \item Erweiterung auf \textbf{33} Basis-Konfigurationen
            \item Vermeidung topologischer Fehler an Zellgrenzen durch konsistente Kantentriangulierung
            \item korrekt rekonstruierte Isoflächen
            \item Einsatz des \textbf{Asymptotic Decider} zur Auflösung mehrdeutiger Fälle
            \end{itemize}
          \end{minipage}
        }
      }
    \end{column}

    \begin{column}{5cm}
      \includegraphics[scale=0.13]{Images/MC33.png}
      \vspace{0.3em}
    \end{column}
  \end{columns}
\end{frame}
\end{comment}

\begin{frame}[t]
  \frametitle{Dual Marching Cubes}
  \begin{itemize}
    \item Pro Zelle im Octree ein Vertex im Dualgitter
    \item Vertex durch Minimierung einer QEF bestimmt
    \item QEF basiert auf Tangentialebenen
\end{itemize}
\[
  \text{Tangentialebene: } T_i(x) = \nabla f(p_i) \cdot (x - p_i)
\]
\[
  \text{Fehlerfunktion: } E(w, x, y, z) = \sum_i\frac{(w - T_i(x, y, z))^2}{1 + ||\nabla f(p_i)||^2}
\]
\includegraphics[scale=0.3]{Images/horse.png}~\scriptsize[10]
\hspace{1cm}
\includegraphics[scale=0.2]{Images/DualMarchingCubes.png}~\scriptsize[11]
\end{frame}

\begin{frame}[t]
  \frametitle{Neural Marching Cubes}
  \noindent
  \makebox[\linewidth]{%
    \includegraphics[scale=0.1]{Images/teaser.png}~\scriptsize[11]
    \hfill%
  }
  \begin{itemize}
    \item neuronales Netz
    \item geometrische Merkmale (scharfe Kanten und Ecken)
  \end{itemize}
\end{frame}

% Zusammenfassung
\section{Zusammenfassung}
\begin{frame}[t]
  \frametitle{Zusammenfassung Marching Cubes}
  \begin{itemize}
    \item Effizienter Algorithmus zur Isoflächen-Erzeugung aus volumetrischen Daten
    \item Einfach, schnell, weit verbreitet % in Medizin, Simulation und Grafik
    \item Herausforderungen:
    \begin{itemize}
      \item Ambiguitäten (topologische Inkonsistenzen)
    \end{itemize}
    \item Normale: aus Gradienten der Skalarwerte berechnet
    \begin{itemize}
      \item z.B. zentrale Differenzen
      \end{itemize}
      \vspace*{-0.3cm}
      \noindent\hspace*{-1em}\rule{\paperwidth}{0.4pt}
    \item Erweiterungen:
    \begin{itemize}
      \item \textbf{Asymptotic Decider}: vermeidet Ambiguitäten
      \item \textbf{Marching Tetrahedra}: vermeidet Ambiguitäten
      \item \textbf{MC 33}: topologisch korrekt
      \item \textbf{Dual MC}: bessere Meshqualität (Quads)
      \item \textbf{Neural MC}: lernbasierte Oberflächenapproximation
    \end{itemize}
  \end{itemize}
\end{frame}

\begin{frame}[t]
\frametitle{Quellen}
\scriptsize
\begin{itemize}
\item Lorensen, W. E.; Cline, H. E. \emph{Marching Cubes: A High Resolution 3D Surface Construction Algorithm}. ACM SIGGRAPH Computer Graphics, 21(4):163-169, 1987.
\item Schaefer S.; Warren J., \emph{Dual Marching Cubes: Primal Contouring of Dual Grids}, IEEE Xplore, 2004.
\item Chen, Z.; Zhang, H. \emph{Neural Marching Cubes}. ACM Transactions on Graphics (SIGGRAPH Asia), 40(6), 2021.
\item Chernyaev, E. \emph{Marching Cubes 33: Construction of Topologically Correct Isosurfaces}
\item Wikipedia-Autoren. \emph{Marching Cubes}. Wikipedia – Die freie Enzyklopädie, Artikelversion vom 19. Juni 2025.
\item Wikipedia-Autoren. \emph{Marching Tetrahedra}. Wikipedia – Die freie Enzyklopädie, Artikelversion vom 27. Juni 2025.\\
\item Nielson, G. M.; Hamann, B. \emph{The Asymptotic Decider: Resolving the Ambiguity in Marching Cubes}. Proceedings of the 2nd Conference on Visualization (VIS '91), S. 83–91, 1991.
\end{itemize}
\end{frame}

\begin{frame}[t]
\frametitle{Bildquellen}
\scriptsize
\begin{itemize}
\item[{[1]}] Carleton College Computer Science, \emph{Smoothing}, Shape Modeling Assignment, 2004–2005.  
\item[{[2]}] P. Baxa, V. Skala, R. Mouček, \emph{An Error Estimation for Isosurfaces}, EDU+COMPUGRAPHICS '97, Algarve, Portugal, December 1997.  
\item[{[3]}] C. Oltmann, "Voxel Terrain w/ Marching Cubes", YouTube, May 2021
\item[{[4]}] O. Alexandrov, \emph{Illustration of torus}, Wikimedia Commons, 13 July 2008. [Online]
\item[{[5]}] Long, Z.; Nagamune, K. \emph{A Marching Cubes Algorithm: Application for Three-dimensional Surface Reconstruction Based on Endoscope and Optical Fiber}. International Journal on Information 18(4):1425-1437, 2015
\item[{[6]}] Wikimedia Commons. \emph{Tetrahedral Cases Visualization}, 2011.
\item[{[7]}] W.C. Yang, "Fluid Simulation: Level Set \& Marching Cube", Computer Graphics / Fluid Simulation, July 2019.
\item[{[8]}] M. Jahns, \emph{Wie lässt sich voxelbasierte 3D-Umgebung polygonisieren?}, Bachelorarbeit, Freie Universität Berlin, Fachbereich Mathematik und Informatik
\item[{[9]}] Schaefer S.; Warren J., \emph{Dual Marching Cubes: Primal Contouring of Dual Grids}, IEEE Xplore, 2004.
\item[{[10]}] Schaefer S.; Warren J., \emph{Dual Marching Cubes: Primal Contouring of Dual Grids}, IEEE Xplore, 2004.
\item[{[11]}] Chen, Z.; Zhang, H. \emph{Neural Marching Cubes}. ACM Transactions on Graphics (SIGGRAPH Asia), 40(6), 2021.
\end{itemize}
\end{frame}

\end{document}
